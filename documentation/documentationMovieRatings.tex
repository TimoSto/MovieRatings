\documentclass[12pt]{article}

%Allgemeine Einstellungen

%Abstände
\usepackage[a4paper,left=2.5cm,right=2.5cm,top=2cm,bottom=3.5cm,headsep=12pt,footskip=24pt]{geometry}%Bottom extra 0.5cm für Footer

%Deutsches Sprachpacket
\usepackage[german,ngerman]{babel}

%Times New Roman
\usepackage{mathptmx}

%Titelseite einbinden
\usepackage{pdfpages}

%1.5-Zeilenabstand
\usepackage[onehalfspacing]{setspace}

%Stil der Überschriften, siehe ueberschriften.sty
\usepackage[numeric]{ueberschriften}

%Stil des Inhaltsverzeichnisses, siehe inhaltsverzeichnis.sty
\usepackage[numeric]{inhaltsverzeichnis}

%Abkürzungsverzeichnis, siehe abk_verzeichnis.sty
\usepackage{abk_verzeichnis}

%Stil der Fußzeilen, siehe fusszeilen.sty
\usepackage{fusszeilen}

%Literaturverzeichnis und Zitate, siehe literatur.sty
\usepackage{literatur}

%Stil für Header und Footer, siehe header_footer.sty
%Wenn nicht erwünscht, müssen auch die Befehle \frontmatter, \mainmatter auskommentiert werden
\usepackage{header_footer}

%Stile für Code-Ausschnitte, siehe codes.sty
\usepackage{codes}

%Stile für Anhänge, Bilder, ...
\usepackage{anhang}

%Silbentrennung (manche Worte werden am Zeilenende nicht getrennt, diese müssen dann nachgetragen werden)
\usepackage[T1]{fontenc}
\hyphenation{öf-fent-lich-en}

\makeatletter
\def\gnewcommand{\g@star@or@long\new@command}
\def\grenewcommand{\g@star@or@long\renew@command}
\def\g@star@or@long#1{% 
  \@ifstar{\let\l@ngrel@x\global#1}{\def\l@ngrel@x{\long\global}#1}}
\makeatother

%DEBUGGING (Zeigt Boxen an)
%\usepackage{showframe}

\begin{document}

\renewcommand{\mytitle}{Ausarbeitung eines Dashboards mit PowerBI zum Vergleich\\der Corona-Zahlen mit den wöchentlichen TMDb-Trends}%Titel für oben links
\renewcommand{\myauthor}{Vanessa Kriebel, Julia Groeniger, Paul Schäfer,\\Timo Stovermann, Bastian Wynk}%Name für unten links
\renewcommand{\headheight}{40pt}%Bei Mehrzeiligem Titel muss Headerhöhe angepasst werden

\includepdf[pages={1-}]{titelseite.pdf}

\frontmatter%Stil des Headers/Footers ändern

\pagenumbering{Roman}

\addcontentsline{toc}{part}{Abkürzungsverzeichnis}%Abk-Verz. ins Inhaltsverzeichnis
\printabbreviations%abk_verzeichnis.sty
\clearpage
\renewcommand{\plaintitle}{Abbildungsverzeichnis}
\addcontentsline{toc}{part}{Abbildungsverzeichnis}
{\def\makebox[#1][#2]#3{#3}%
\listoffigures
}
\clearpage
\renewcommand{\plaintitle}{Tabellenverzeichnis}
\addcontentsline{toc}{part}{Tabellenverzeichnis}
{\def\makebox[#1][#2]#3{#3}%
\listoftables
}
\clearpage
\renewcommand{\plaintitle}{Inhaltsverzeichnis}%Titel für oben Rechts
%Defbox, damit gepunktete Linie bis zur Zahl geht
{\def\makebox[#1][#2]#3{#3}%
	\tableofcontents
}

\addtocontents{toc}{\vspace{24pt}}%Freiraum im ToC

\clearpage
\mainmatter%Stil des Headers/Footers ändern
\pagenumbering{arabic}

\part{Einführung}

\part{Werkzeuge}

\part{Datenbasis}
Für diese Dashboard-Anwendung wurde über mehrere Wochen hinweg eine Datenbasis aufgebaut, um
\section{Datenquelle}
Die verwendeten Daten stammen von der TMDb-API. Dies ist eine REST-API, über welche Informationen zu Filmen, Serien, Personen und Trends ermittelt werden können. Ein beispielhafter Request könnte folgendermaßen aussehen:
\begin{center}
GET \textit{https://api.themoviedb.org/3/movie/popular?api{\_}key=<api{\_}key>}
\end{center}
Der \textit{api{\_}key} ist für jeden Account unterschiedlich. Die TMDb-API liefert auf einen solchen Request ein JSON-Objekt aus. In diesem Fall beinhaltet diese Antwort die 20 aktuell populärsten Filme.
\part{Dashboard}

\end{document}